
\documentclass[11pt,a4paper]{ivoatex/ivoa}
\input ivoatex/tthdefs

\title{Radio Astronomy in the VO:\\ implementation note} 

% see ivoatexDoc for what group names to use here
\ivoagroup{Radio Astronomy interest group}

\author[https://wiki.ivoa.net/twiki/bin/view/IVOA/HarrisonBarlow]{Harrison Barlow} 
\author[https://wiki.ivoa.net/twiki/bin/view/IVOA/FrancoisBonnarel]{Fran\c cois Bonnarel}
\author[https://wiki.ivoa.net/twiki/bin/view/IVOA/BaptisteCecconi]{Baptiste Cecconi}
\author[https://wiki.ivoa.net/twiki/bin/view/IVOA/JamesDempsey]{James Dempsey}
\author[]{Vincenzo Galluzzi}
\author[https://wiki.ivoa.net/twiki/bin/view/IVOA/SeverinGaudet]{Severin Gaudet}
\author[https://wiki.ivoa.net/twiki/bin/view/IVOA/YanGrange]{Yan Grange}
\author[https://wiki.ivoa.net/twiki/bin/view/IVOA/PaulHarrison]{Paul Harrison}
\author[https://wiki.ivoa.net/twiki/bin/view/IVOA/MarkKettenis]{Mark Kettenis
}
\author[https://wiki.ivoa.net/twiki/bin/view/IVOA/MarkLacy]{Mark Lacy}
\author[https://wiki.ivoa.net/twiki/bin/view/IVOA/AlanLoh]{Alan Loh}
\author[https://wiki.ivoa.net/twiki/bin/view/IVOA/FelixStoehr]{Felix Stoehr}
\author[https://wiki.ivoa.net/twiki/bin/view/IVOA/AlessandraZanichelli]{Alessandra Zanichelli}
\author{and the RadioAstronomy Interest Group}

\editor{Mark Lacy, Fran\c cois Bonnarel}

% \previousversion[????URL????]{????Concise Document Label????}
\previousversion{This is the first public release}


\begin{document}
\begin{abstract}
	This document provides some hints to help Radio astronomy projects data provider to build VO
	services on top of their archives. It gathers experience gained by various groups around the world.
\end{abstract}


\section*{Acknowledgments}

Authors gratefully thank all the radioastronomy projects who provided story-telling gathered in 
Appendix.

\section*{Conformance-related definitions}

The words ``MUST'', ``SHALL'', ``SHOULD'', ``MAY'', ``RECOMMENDED'', and
``OPTIONAL'' (in upper or lower case) used in this document are to be
interpreted as described in IETF standard RFC2119 \citep{std:RFC2119}.

The \emph{Virtual Observatory (VO)} is a
general term for a collection of federated resources that can be used
to conduct astronomical research, education, and outreach.
The \href{https://www.ivoa.net}{International
Virtual Observatory Alliance (IVOA)} is a global
collaboration of separately funded projects to develop standards and
infrastructure that enable VO applications.


\section{Introduction}

Radio astronomy data were considered in the first decade of IVOA, within the Characterisation data 
model \citep{2008ivoa.spec.0325L} and the first version Simple Image Access (SIA1, \cite{std:SIAP}). 
When multidimensional data science became a CSP priority (2013), this drove the emergence of new 
protocols that included treatment of polarisation and image cubes, both important for radio 
astronomy: ObsCore \citep{std:OBSCORE}, SIA2 \citep{std:SIAV2} , DataLink \citep{2015ivoa.spec.0617D} 
and SODA \citep{std:SODA}. 

Since that time, radio astronomy projects have further increased in size and variety, necessitating 
a review of how the VO can better work with radio data. In this document, we begin this process as 
follows:

\begin{itemize}
\item  In Sections 2 and 3 below, we give a list of existing radio astronomy VO services with protocol 
used for ALMA (section~\ref{sec:ALMA}), LOFAR (section~\ref{sec:Astron}), ASKAP, ATCA,
(section~\ref{sec:CASDA}) NRAO and CGPS (section~\ref{sec:CADC}) and e-MERLIN (section~\ref{sec:emerlin}).
The content of these Sections is organised  by category of data and then type of services. Section 2 
presents science data, Section 3 concentrates on raw (visibility) data. We try to extract some generic 
lessons on how services can proceed to build services.
\item We asked projects to provide us some ``story-telling" of what they have done (and how they did 
it) : this can be found in the appendix
\end{itemize}

The scope of this document is restricted to data from heterodyne receivers working at frequencies of 
0-1\,THz. Far-infrared data has a lot of data from space missions, and often makes use of bolometer 
technology, which has a lot more in common with optical/near-infrared data than most radio data.

\section{Science data}

In this section we only consider high level data products that can be produced from any of 
interferometers, single dishes etc. We defer discussion of observational data to Section 3.   



\subsection{Radio 2D images and cubes}
\subsubsection{Data Access Layer (DAL) services}
The combination of the ObsCore data model and the VO's Table Access Procotol (TAP; \cite{std:TAP}), 
known as ObsTAP, and the SIA are classical "discovery, then access" services adapted to images and 
cubes, and can be used to serve radio image products. Metadata characterising the observation needed for 
ObsCore can be extracted from FITS headers or observation logs. The main issues for metadata extraction 
are to choose the granularity and quality of datasets to expose, and to find solutions for access in 
cases where the datasets are huge. ObsCore tables can be served by two different interfaces, TAP  
(queryable via Astronomy Data Query Language (ADQL) \citep{2008ivoa.spec.1030O} ), or SIA2 (a parameter 
based interface). The SIA1 interface remains a VO standard. It is a parameter based interface with a 
different set of parameters than SIA2. The standardised SIA1 response is not based on ObsCore and has 
archaic features. The SIA1 interface should be considered as a legacy standard for the first-generation 
of VO services. We do not recommend its implementation for new services. 

For solar system data EPN-TAP is an equivalent to ObsTAP which implements some specificities required for this kind of data (sections~\ref{sec:maser} and~\ref{sec:Nancay}). 

For large datasets, where retrieval of entire images/cubes may be cumbersome,  DataLink and SODA are 
appropriate solutions that have been experimented with at CADC (section~\ref{sec:CADC}) and by CASDA. 
SODA allows the extraction of sub-cubes by selecting a smaller spatial coverage or spectral range. 
DataLink enables the linkage of a SODA interface to a specific dataset discovered with ObsTAP or SIA. 
Access to raw data and provenance information can also be provided via DataLink, as it is the case for 
INAF radio archive and ALMA science archive.

The Common Archive Observation Model (CAOM) may be a good intermediary between the set of metadata in 
image headers metadata and VO standards. Some CAOM standard metadata map easily to ObsCore attributes 
while others can be used to help develop SODA or DataLink services. This is the way CADC implements DAL 
services. DACHS \footnote{\url{http://docs.g-vo.org/DaCHS/}} is a complete python toolkit allowing to 
create all DAL services and register them. Data centres such as CDS, CSIRO and CADC have developed their 
own open java libraries (vollt, CASDA VO Tools, openCADC). These libraries can be used to develop new 
services.


\subsubsection{HiPS}
A completely different way to make image data available through the VO is by using HiPS 
\citep{2017ivoa.spec.0519F}. HiPS provides a hierarchical progressive access to pixels. It can be used 
for 2D images and 3D cubes. In the radio domain, it is often used for 2D images produced from spectral 
cubes. The ``tile progenitor''  functionality may be used to help discover original images or cubes. 
ASTRON has published HiPS for Apertif and LOFAR data, and CADC for CGPS(section~\ref{sec:CADC}). Several 
software packages are able to produce HiPS images, and when that is done no more software is needed to 
install a HiPS server - a simple http server allows access to the hierarchical data. There are several 
VO applications able to visualise HiPS (AladinLite, Aladin Desktop, Stellarium, etc...). 

\subsubsection{List of working services}
\begin{itemize}
\item ObsTAP services: ALMA(section~\ref{sec:ALMA}), CADC (VLA Sky Survey, Canadian Galactic Plane 
Survey -DRAO-, other DRAO data, VLA Galactic Plane Survey, Section~\ref{sec:CADC}), CSIRO (CASDA, ATOA - 
Section~\ref{sec:CASDA}), ASTRON (section~\ref{sec:Astron}).
\item SIA1: CADC, Astron (APERTIF,LOTSS), Skyview/GSFC (FIRST, NVSS)
\item SIA2: ALMA, CADC, CASDA, ASTRON (APERTIF, LOTSS)
\item DataLink: ALMA, CADC, CASDA
\item SODA: CADC, CASDA
\item HiPS: CASDA, Apertif, LOTSS, CGPS at CADC
\item Applications: Aladin, CASSIS,  DS9, TOPCAT allow discovery and simple visualisation of cubes. Neither the 
current CASA viewer, nor the new CARTA viewer support VO searches or SAMP \citep{2009ivoa.spec.0421B}, 
though CARTA has long-term plans to do so.
\end{itemize}


\subsection{Polarization data}
ObsCore attributes pol\_states and pol\_xel allow to describe the polarization content of datasets. 
Polarization add a 4th dimension to spectral cubes, but this does not prevent ObsCore from describing 
them. Several of the ObsTAP and SIA services listed above provide Polarization data (eg: Astron, 
CADC...)


\subsection{Radial velocity, rotation measure and velocity dispersion maps}
These are actually classical 2D maps derived from spectral cubes, as are the continuum, line or "moment 
zero" maps, except that the Observable quantity is not a flux density. These are thus manageable with 
an ObsCore/SIA service. The provider only has to choose the best o\_ucd value, for example:

\begin{itemize}
    \item rotation measure: phys.polarization.rotMeasure;instr.rmsf
    \item radial velocities : spect.dopplerVeloc
    \item velocity dispersion : phys.veloc.dispersion
\end{itemize}

\subsection{Source catalogues}
Extraction of radio sources from survey images generates catalogues of sources. Their provenance is 
specific, but otherwise they are no different from other type of catalogues that can be served by many 
VO services, following either SCS \citep{std:SCS} or TAP standards. A few examples are:
\begin{itemize}
\item Simple Cone Search (SCS) services: CASDA, ATOA, PSRDA (Parkes Pulsar data archive) - see 
section~\ref{sec:CASDA} - , CDS/VizieR
\item TAP services: CDS/Vizier (NVSS, FIRST, etc...), ASTRON (WSRT, TGSS, LoFAR), CASDA, ATOA, PSRDA 
\end{itemize}


\subsection{TimeSeries, dynamic spectra, pulsar data}

Data sampled on rapid (millisecond) timescales are common in pulsar astronomy and the study of Fast 
Radio Bursts (FRBs). 

Solar system objects (the Sun, and some planets, such as Jupiter), also can show 
rapid time variability in their radio emission. Data streams typically capture flux variations along 
with frequency and polarisation information, in the form of ``dynamic spectra".

These products may be described with ObsCore and exposed through TAP or, in some cases, SIA. 
Dataproduct\_type ``timeseries'' is ambiguous for dynamic spectra. The term ``dynamic-spectrum'' 
has been recently added in the draft \emph{Data Product Type}\footnote{\url{https://www.ivoa.net/rdf/product-type/2021-06-21/product-type.html}}
vocabulary (after a science case driven by EPN-TAP services and Pulsar data), and should be preferred in this case. 
Such data are made available through JIVE 
(section~\ref{sec:JIVE}) and Nan\c cay (section~\ref{sec:Nancay}) services and also by the Parkes Pulsar 
Data Archive (PSRDA).

\subsection{Using MOC for discovery}

The Multi-Order Coverage (MOC; \cite{2019ivoa.spec.1007F}, spatial MOC) standard enables the description 
of the coverage of surveys or data collections on the sky using the Healpix tessellation. The IVOA 
registry integrates these survey MOCs. This is very useful for all sky discovery of the radio data 
collections. CADC, CASDA, ASTRON, .. provide MOCs for their data embedded in VO services. An extension 
of MOC to Time (combined or separated from space) will be standardised by next version of the MOC 
specification currently in discussion.

\subsection{Registration} 
To be discoverable in the VO, a service must be registered. When a registered VO service is specific to 
some radio data collection it's easy to discover it. When a generic service contains radio data as well 
as other types, it may be a good idea to register the service with a specific auxiliary capability, as 
explained in the IVOA endorsed note discovering data collections \citep{2019ivoa.rept.0520D}.

\section{Raw data}
The nature and format of raw radio astronomical data 
depends on the observational technique. Interferometric telescopes typically store data in the form of 
complex visibilities (post correlation), while single dish or beam-forming observations will provide on-
the-fly maps or other kind of mapping data, but also uncalibrated flux variation as a function of 
position, time and frequency. 

The IVOA does not promote specific formats. Visibilities may be provided as Measurement Sets, FITS-IDI, 
RPFITS or UVFITS. Pulsar data are provided in PSRFITS and single dish data may use standard FITS, 
MBFITS, SDFITS or HDF5. An 
important feature of IVOA is to be able to describe the media type (format) of the data before 
retrieval. Extensions of the mime type vocabulary may be standardised in a near future.  

In order to foster interoperability, consistent and eventually convergence, metadata mapping between
the metadata of the community products and the IVOA models should be provided. The task should
start with the various flavours of FITS described above, and should follow with the HDF5 flavours\footnote{See here for updated details: \url{https://www.astron.nl/lofarwiki/doku.php?id=public:documents:lofar_documents}} 
\citep[see, e.g., the LOFAR HDF5 implementation, ][]{2011ASPC..442...53A}.

\subsection{Interferometry}
Complex visibilities can be discovered using three different services according to their relationship 
with science data and to their characterisation: SCS, ObsTAP and DataLink. SCS is adapted to datasets 
with small field of view, where a pointed observation can be
represented by an entry in a catalog. ObsTAP is well adapted for visibility data matching the science 
spectral cube contours derived from them. Special attention may be needed to describe interferometry 
data for observations containing several targets and various spectral resolutions. So far JIVE has 
experimented with splitting of an observation into several datasets for the discovery but this may 
pose challenges in terms of unique identification of data. DataLink exposition allows 
the linking of visibility data to the science data they produce but makes direct discovery more 
difficult. 
\begin{itemize}
\item ObsTAP services: ATOA, MWRA (section~\ref{sec:MWA}), ASKAP, JIVE (section~\ref{sec:JIVE})
\item DataLink : ALMA, ASTRON (Apertif)
\item SCS : ASTRON (Apertif)
\end{itemize}
\subsection{Single dish data}
Before being used to produce science data in a more classical format (nD cubes and images, dynamic
spectra or spectra),  single dish  raw or calibrated data are generally stored or transported as 
collections of spectra for a given position, time, polarisation state, as it is the case eg in the 
SDFITS standard. Such datasets could be accessed from science data using DataLink but can also be 
discovered in an ObsCore service. In the latter case dataproduct\_type, dataproduct\_subtype, 
calibration level and access\_format should be adapted to single dish specificities.
\begin{itemize}
\item ATOA 
\item INAF (section~\ref{sec:INAF})
\item Other candidates for VO integration
\begin{itemize}
\item  FAST
\item Arecibo
%\item ???
\end{itemize}
\end{itemize}
\subsection{Beam forming data}
Some telescopes arrays (LOFAR, NenuFAR, VLBI) may provide "beam-forming data" by coherently summing up 
signals from the different antennae, instead of registering the visibilities of the interference 
patterns. In this way they obtain time/frequency data. At the time of writing these are the projects 
providing such kind of data:
\begin{itemize}
\item JIVE
\item Nan\c cay
%\item ...
\end{itemize}
\section{Provenance}

IVOA has specified the Provenance DataModel \citep{2020ivoa.spec.0411S} which allows the history of the 
data to be traced via structured metadata. The Provenance model enables high-level data products to be 
linked the raw data, and the processing and QA steps to be identified. At the time of writing there are 
no radioastronomy VO service providing IVOA provenance metadata, though a service is planned by the 
Spanish VO.   

%\bibliographystyle{unsrt}
\bibliography{references}


\appendix
\section{ Current Implementation stories}

\subsection{ALMA}
\label{sec:ALMA}
%from Felix
The ALMA archive has implemented ObsTAP (\url{https://almascience.org/tap}), SIAv2 (\url{https://
almascience.org/siav2}) and DataLink (\url{https://almascience.org/datalink}) services.
The datalink service offers the same files that are also offered through the normal download, including 
visibility data. The searches, however, happen on a higher-level of the data where metadata from several 
sets of visibilities is combined.

The decision of which granularity of the very hierarchical ALMA data
structure to offer took quite some time. In the end we opted for serving
one row corresponding to a project\_code$+$memberObsUnitSet$+$SpectralWindow. This roughly relates to 
one row per primary FITS product that can be created.

A particular difficulty for ALMA which is not yet entirely resolved is that three identical services are 
running at the three ALMA regional centres where the URLs like \url{https://almascience.org/tap} get 
redirected
to the closest regional centre e.g. \url{https://almascience.eso.org/tap}.
Whether it is best to register the three TAP, three SIAv2 and three DataLink services or only 
registering the original link is not clear and depends also on how the VO tools handle such cases.

For the downloads through DataLink there are at least four hierarchy
levels of the data that people could be interested in accessing (Project, ObsUnitSet, Source or Spectral
Window). We hope that through astroquery those can be made available to the users. The download of
proprietary data through astroquery and DataLink is in the works.

\subsection{ASTRON}\label{sec:Astron}
The Netherlands Institute for Radio Astronomy (ASTRON) is responsible for the operations of several radio astronomical facilities. The major surveys are and will be made accessible through the   {\href{https://vo.astron.nl/}{ASTRON VO service}}, running an instance of of DaCHS \citep{10.1016/j.ascom.2014.08.003}. Also, HiPS data is being offered through \href{https://hips.astron.nl/}{another service}. Currently, those services contain data from the Westerbork Synthesis Radio Telescope (WSRT), the Apertif system, the LOw Frequency ARray (LOFAR). Also, the Alternative Data Release of the the TIFR GMRT Sky Survey (TGSSADR), which is used as a sky model for LOFAR observations, is served through the ASTRON VO.

Currently, all releases of data through the ASTRON VO have been of imaging surveys (like Apertf DR1 or LoTSS DR1). This data can be easily shared using the existing VO standards like SIAP2, TAP and ObsCore. Also since the image data format generally is FITS, the metadata gathering can be done in the same way as one would do it for other wavelengths. However, due to the fact that single images are taking up a lot of disk space, storage at an external location (e.g. a national supercompute centre) may be needed. We have developed tooling to only download the header from a remotely saved FITS file, and save it to csv format, so that not all data needs to be downloaded just to read their header. The csv files can then be used as input for the metadata gathering. For access to the data, DataLinks link to the remote resources. Also, we are investigating if this can be combined with using SODA cutouts to minimise the amount of data that is actually shipped to the end users.  

One of the forthcoming data releases is that of the Apertif FRB data (ARTS-DR1). It is to be expected that more releases of timing data (pulars, FRBs, etc) will follow. Futhermore, LOFAR data from solar observations are candidates to be made public, which would require using of the appropriate standards for solar-system objects. 

Visibilities are currently being shared as part of the metadata, as part of the provenance through DataLinks. Since all visibilities are stored on tapes, access to them cannot be granted sychronously. The current implementation uses a web page showing relevant information, with hints to users on how to request access through the helpdesk. For future data releases, we consider developing a UWS staging service that can handle the process of bringing data online. Even though directly querying visibilities from LOFAR or Apertif in the VO is not currently planned, ASTRON is involved in the development of visibilities standards, which could be used for specific data collections that may be released in the future. 

\subsection{CADC}
\label{sec:CADC}
%Severin's input
because we use the CAOM data model for all collections, implementation has been at 3 levels:
CAOM workshops have made a point of involving people working with radio and millimetre data so their 
input on supporting radio data use cases drove some of the evolution of the CAOM data model (currently 
at version 2.4).

CADC has always been involved in the IVOA and has taken on the effort to support VO protocols and models 
within the core of the CADC software architecture. In that sense, all CADC users use "the VO".  An 
example: if a data centre is building a query page, build it with a TAP backend.

The data engineering work for a new collection is third type of implementation. This is the effort to 
map a collection's metadata into CAOM2 concepts. That involves working with science domain experts both 
at the source of the data and at CADC, and with developers that translate the mapping into code 
(pipelines) built from common libraries and customised for the collection. These pipelines create and 
persist CAOM instances from metadata databased and/or from files.

\subsection{CDS}
\label{sec:CDS}
CDS is not an observatory archive. It's a value added data center aiming at distribute published data. Related to this data center activity CDS also developped a strong expertise in VO standards with participation to their creation and evolution.  

\subsubsection{VizieR and SPECFIND}

VizieR is a huge collection of astronomical  catalogues associated with publications. At the time of writing VizieR gives access to more than 2900 catalogues containing radiosources or radio measurements with a complement of more thna 200 milletric . It also provides access to radio images and cubes associated to publications. IN addtion, since 2005 CDS was able to cross-identify 1600000 radio sources among 204 tables and to provide recosntructed spectra for 340000 of tem. SPECFIND 3.0 has been published in 2021 \citep{2020yCat.8104....0S}   


\subsubsection{HiPS}
CDS is not storing original images and cubes from observatory archives. However it is storing  a large collection of image, catalogue and cube HiPS. Dozens of them belong to the radio domain, from Effelsberg to SCUBA and from LOFAR to Apertif. HiPS is both an access and visualisation facility for pixel and catalogue data as well as it can be a discovery tool for original data. 

\subsubsection{Old and recent contribution to standardization useful for radio data: visibility data ObsTAP testbed and ObsCore extension }

CDS participated to the development of many standards useful for discovery and access of radio data such as VOTable, SAMP, Characterisation DM and Obscore, SIA 2.0, SODA, DataLink, HiPS and MOC. 
Recently discovery of visibility data became a strong requirement of the community. Within the scope of the european ESCAPE project, CDS experimented the development of a testbed service exposing some visibility measurement sets as a set of consistent ObsCore datasets by separating the different fields, spectral ranges and spectral resolutions present in each observation \citep{2020arXiv201209273L}. The limits encountered in this exercice were a strong driver to start defining the ObsCore extension for  visibility data specifications.


\subsection{CSIRO Radio Archives}\label{sec:CASDA}

\subsubsection{CASDA}

The CSIRO ASKAP Science Data Archive (CASDA) aims to provide science ready data products from the Australian Square Kilometre Array Pathfinder (ASKAP) telescope. 
We use FITS format for images, cubes, spectra and moment maps, and VOTable format for Catalogue data. 
In addition we provide validation data-products as tar files and calibrated visibilities for some observations in tarred CASA measurement set format.

We provide three means of discovering and accessing data, a) via the website \footnote{https://data.csiro.au}, b) using VO protocols and c) using the astroquery.casda module. 
Discovery for the web site uses an ElasticSearch index while the TAP, SIAP and SSAP services query our PostgreSQL metadata database. 
Our Implementation of these protocols is available publicly as CASDA VO Tools \footnote{https://doi.org/10.25919/zy6w-h884}.
All science data products are listed in the ObsCore view, however metadata for visibilities is limited. 
Data access for all three access paths goes through the same module implementing the Datalink and SODA protocols. 
Currently all data are held on tape and, if not already online, are recalled on demand to disc as part of the data access process. 
We are working to move to an object store and have all data online.

During deposit of data products we extract metadata from FITS and VOTable files including position and observation information.
These are then stored in a PostgreSQL database which is directly queried by the VO services and is used to populate the ElasticSearch index.
As part of the processing, validation reports are produced for each observation including quality metrics.
These can be viewed on the website and queried via TAP.su
The science teams are also validating the data products and providing quality ratings and notes.
These can be accessed via the website and the quality ratings can be found in all data-products in their ObsCore listing.

\subsubsection{PSRDA}

The Pulsar Data Archive (PSRDA) provides pulsar observations taken predominantly at the Parkes 64m radio telescope.
All data are in PSRFITS format and data include both search and fold mode observations.
The index of observations may be queried via TAP. Data access is only provided via the website \footnote{https://data.csiro.au}.

During deposit, metadata such as target, central frequency, the frontend and backend used, and the duration of the observation are extracted from the FITS headers and stored in a PostgreSQL database.
This is indexed by ElasticSearch for use in queries on the website.
Currently all data are held on tape and, if not already online, are recalled on demand to disc as part of the data access process.
Future work will bring all data online via an object store.

\subsubsection{ATOA}

The Australia Telescope Online Archive (ATOA) serves spectral line observations from both the Australia Telescope Compact Array (ATCA) and Parkes.
Most ATCA data is in in the form of visibilities in RPFITS format, with files split by time and thus containing multiple sources and spectral windows.
Older spectral line Parkes data are in SDFITS format while data from the latest Parkes receivers are in HDF5 format with metadata based on SDFITS.

Metadata from the RPFITS, SDFITS and HDF5 files are retrieved as part of deposit processing and stored in a MySQL database. Web searches on the ATOA site \footnote{https://atoa.atnf.csiro.au/} use this database.
A copy of the metadata is made to a PostgreSQL database where it is made available for TAP queries against an ObsCore schema. 
All data are online and available to both web and VO users for download.

\subsection{e-MERLIN}
\label{sec:emerlin}
The staff of e-MERLIN telescope network (\url{http://www.e-merlin.ac.uk}) and its predecessor MERLIN have had a long association with the virtual
observatory via involement with the AstroGrid project \citep{2002ASPC..281....3M}. An early service that
 provided `on-demand' images from visibility data was described in a paper somewhat optimistically entitled
``Painless Access to Interferometry Images Comes Closer'' \citep{2006ASPC..351..267R}. This service pre-dated standards
such as SODA which might have been considered for use in creating such a service - although SODA in itself does not
have sufficient power to express all of the parameterization that service provided. Unfortunately, with the cessation
of funding of AstroGrid in 2009, further development of the software suite also stopped. The main effort in e-MERLIN was 
put into providing a processing pipeline \footnote{\url{https://github.com/e-merlin/eMERLIN_CASA_pipeline}} and archive data access to
PIs using non-VO compliant methods. Recently, however, work has restarted on adding modern VO compatibility to e-MERLIN
archive access as a pilot for the UK SKA regional centre \footnote{\url{https://astronomers.skatelescope.org/the-ska-regional-centres/}}.
The source code for the new VO capabilities derives in part from the original AstroGrid source and is being managed under the
``Javastro'' GitHub project \url{https://github.com/Javastro}.

\subsection{INAF Radio Data}
\label{sec:INAF}
%from Alessandra
INAF manages three fully steerable reflector antennas for radio
astronomy: the 32m dishes at Medicina and Noto and the 64m
Sardinia Radio Telescope (SRT). These observing facilities can be
used separately as single dish instruments (SD) or in a coordinated
national network, sometimes joined by other EVN facilities, that
delivers VLBI interferometric data. Pulsar observations are supported 
in both single-dish and interferometric mode. The various
raw data types are characterized by different output data formats:
FITS, FITS-IDI, PSRFITS and FILTERBANK. Ancillary information
contained in the observing schedules and telescopes/correlator
logs is archived in separate files for each dataset. The INAF radio
archive stores continuum and spectropolarimetric raw data from
single dish, pulsar and VLBI observations. Storage of processed
data is planned once the Archive Science Gateway (including pro-
cessing pipelines, an implementation of VOSpace, etc.) will be
finalised.

To promote the discovery, access and retrieval of data we are
planning to increase the number of adopted IVOA standards or,
at least, solutions that take into account the VO architecture.

Global data discovery and access requires to expose a uniform,
standard data model. To this aim, we have verified that the (internal) 
radio archive data model maps to the ObsCore data model,
at least for what concerns the mandatory elements. Moreover, for
single dish data a comparative analysis between the radio archive
model and the CAOM one has been performed and can potentially
address some specific use cases for provenance, metadata annotation 
and provide the basis for a more general discovery and access
solution.

An ObsTAP service, exposing an ObsCore table, is under
preparation for the holdings of the Radio Archive. We are evaluating 
the usage of the DataLink protocol to access the ancillary
information associated with single dish raw datasets, namely the
telescope log files and schedules. Other information to describe e.g.
the UV coverage of a specific interferometric dataset can be added
in graphical form but currently is yet to be implemented. Another
DataLink use case could be relating additional metadata (for instance 
provenance and data quality metrics) at different stages of
the processing chain.

A more accurate analysis of the Provenance information (and
its capture) needed to document all the phases of data reduction
and assess the data products quality is planned in view of archiving
also processed data.

\subsection{The Joint Institute for VLBI in Europe}
\label{sec:JIVE}
%from Haro Verkouter
The Joint Institute for VLBI in Europe (JIVE) hosts the archive of all radio-astronomical VLBI 
observations of the European VLBI Network (EVN). The datasets are archived as raw interferometric 
visibilities in FITS-IDI format (a registered FITS convention\footnote{\url{ http://fits.gsfc.nasa.gov/
registry/fitsidi.html}}) or in the deprecated UVFITS format for the oldest datasets. The latter format 
is deprecated because it uses random groups and the use of random groups is discouraged in the current
FITS standard. The raw visibilities are the result of processing ("correlation") and should be regarded
as Data Level (DL) 1. The DL0 data, the raw digitised, sampled, voltages from the antennas, is too big to
archive and are deleted after correlation.

Connected to the raw visibilities, the EVN archive stores initial calibration information, flagging 
information, diagnostic plots and the output products of running a generic pipeline. All these artefacts 
are either necessary to create science data products or are a valuable resource in assessing data 
quality and/or scientific potential.

JIVE is working towards making these raw visibilities and support information findable and downloadable 
through ObsTAP and DataLink services; none of the other IVOA services apply naturally to unprocessed 
interferometric data. The pipeline outputs crude images of the observed sources (including the
calibrator(s)). These images will be added as thumbnail/preview images to ObsCore entries through 
DataLink. There is no point in offering these through e.g. SIAP since the images are not at all science 
quality. An important envisaged use case of making the raw visibilities available through VO protocols 
is that this will allow users to find and download historic observations and reprocess them with modern 
software.  A preliminary service that offers ObsTAP and DataLink is up and running but has not been
registered yet.

The FITS-IDI files contain sufficient meta-data to parse almost all the necessary ObsCore parameters 
from there. The raw visibility data sets tend to be large (2020 standards). Typical sizes vary from 
several GB to a few TB per dataset. Radio-interferometric datasets can represent a rather large 
theoretical FoV on the sky. Combined, these properties lead to a large probability of a VO query 
returning GBs of data which may turn out to be unuseable. Two parameters to characterize the useability 
of visibility datasets are proposed to be added to the ObsCore data model such that these parameters can 
be filtered on. The parameters eccentricity and filling factor of the u,v-plane are currently being 
experimented with in ASTRON and JIVE to assess feasibility in computing and usefulness in practice.

It might be possible to create a union s\_region of all the FoVs over all EVN observations in the 
archive and offer that as a MOC, however this is not a primary goal. If time and resources allow it 
might be considered.



\subsection{MASER}
\label{sec:maser}

The MASER (Measurement, Analysis and Simulation of Emissions in the Radio range) is providing the 
community with a science-ready toolbox for low frequency radio astronomy (below 100 MHz). 

The Sun and planets of our Solar System are emitting radio waves through non-thermal radio emission 
processes (plasma instabilities, mode conversion, electrostatic discharges...) and they are the dominant 
radio sources in the lower frequency range (below 50 MHz). Down to 10 MHz ground-based radio telescopes
are used, while space observation are required at lower frequencies. Radio astronomy Solar System 
observations are mostly dynamic spectra, but 2D images, event catalogues and timeseries are also 
applicable. Solar System radio sources are often sporadic and are monitored over long periods of time, 
with continuous or daily observations. 

Solar and planetary radio astronomy data products can be shared through EPN-TAP (combination of EPNcore 
and TAP; \cite{std:EPNTAP}). The goal of EPN-TAP is the same as that of ObsTAP, but the metadata are 
dedicated to solar system observations (e.g., not requiring celestial coordinates, since solar system 
bodies are not fixed in this frame, but relying on the named target objects). EPN-TAP services are 
accessible from any TAP client, but the VESPA query portal (Virtual European Solar and Planetary Access, 
\url{http://vespa.obspm.fr}) allows to search in all registered EPN-TAP services at once.

Accessing data collection spanning over long time periods, and browsing through them is a challenge. 
The space science community has developed several protocols for accessing timeseries or dynamic spectra 
with dedicated server-client architecture. For radio astronomy, Das2 \citep{10.1002/essoar.10500359.1} 
provides the required capabilities, including a server-side resampling option. The latter feature is 
the key here: the Das2 client queries the server with a specific temporal resolution (e.g., its display
resolution). The server sends the data according to this query, with an optimised use of the network
bandwidth. The Das2 endpoints of a given collection can be published accessible in the EPN-TAP service 
response, through the Datalink interface.

The distribution of radio astronomy event catalogues also require specific elements that are not yet
fully available in the IVOA framework. A TFCat (Time-Frequency Catalogue) model and format has been 
developed to support complex spectro-temporal shapes and features. This model is currently being 
tested and has been presented to the IVOA community. 

\begin{itemize}
\item EPN-TAP services: CDPP (AMDA), MASER (Cassini, Wind, ExPRES), Tohoku Univ.\ (Iitate), LOFAR 
(Poland), Nan\c cay (NDA, \emph{NRH, ORFEES})
\item Das2 services: MASER (Cassini, Voyager, ExPRES, \emph{STEREO}), Nan\c cay (NDA, \emph{NenuFAR}), 
Univ.\ Iowa (Galileo, Mars Expres, Juno, Cassini, LWA, Kharkov UTR-2)
\end{itemize}
Note: the services in \emph{italic} are prototypes.

\subsection{The Murchison Wide-Field Array}
\label{sec:MWA}
%from Harrison Barlow
The MWA supports the implementation of various VO standards, which helps to ensure that all data follows 
FAIR access principals. Our usage of some of these protocols are described below:
\begin{itemize}
\item TAP - The MWA ASVO provides an IVOA compliant Table Access Protocol (TAP) service, which can be 
used by any VO/TAP compliant software such as TOPCAT to retrieve MWA observation metadata. Our TAP 
service supports the standard IVOA "ObsCore" schema, in addition to some new MWA-specific schema for 
richer observation metadata. Our implementation is courtesy of CASDA VOTools 
\url{https://github.com/csiro-rds/casda_vo_tools}.
\item VOTable - Users of the MWA ASVO are able to download observation metadata in a VOTable XML format 
which is compatible with widely used astronomical software.
\item Cone Search - The MWA ASVO implements the simple cone search protocol allowing users to quickly
and easily find astronomical sources of interest.
\item ADQL - In addition to our custom-developed search interface, users can also query the MWA TAP
service using Astronomical Data Query Language (ADQL).
\item SSO - While not specific to the IVOA, the MWA ASVO implements a number of different authentication 
methods which are approved by the IVOA. These include logging in with a large number of federated 
identity providers with SAML, as well as oAuth integration with AAO Data Central.
\end{itemize}


\subsection{Nan\c cay}
\label{sec:Nancay}
%Baptiste and Alan input

The Nan\c cay radioastronomy facility is hosting several radio telescopes allowing observation from 
10~MHz to up to several GHz. With the support of PADC (Paris Astronomical Data Centre), the Nan\c cay 
Data Centre (CDN) is publishing radio astronomy data products. The CDN is following the MASER \cite{10.5334/dsj-2020-012}
recommendations for low frequency radio astronomy data distribution and publication. 

At the time of writing of this note, data products from NDA (Nan\c cay Decametric Array) are accessible 
through the VO. Prototypes are under development for the NRH (Nan\c cay Radio Heliograph), ORFEES 
(Observations Radio pour FEDOME et l'Etude des Eruptions Solaires), LOFAR-FR606 (French LFOAR station) 
and NenuFAR (New Extension in Nan\c cay Upgrading LOFAR). Data products from the NRT (Nan\c cay Radio 
Telescope) will be considered later on. 


\subsubsection{NDA}
The NDA is a phased array of 144 helicoidal antennae divided in to sub-arrays of 72 antennae, sensitive 
to either Right-Handed or Left-Handed circular polarisation. This low-frequency radio instrument is
observing the Sun and Jupiter on a daily basis since 1990 and records Routine dynamic spectrum data with 
a spectral range spanning from 10 MHz to 40 MHz and 80 MHz (for Jupiter and Solar observations, 
respectively), a temporal sampling step of 1 second and a frequency sampling of 75 kHz. The daily 
routine observations are published through an EPN-TAP service for data discovery: 
\url{http://vogate.obs-nancay.fr/browse/nda/q}. Three types of products may be retrieved from its 
EPN-TAP service:
\begin{itemize}
\item data files in binary format;
\item Raw data converted to Common Data Format (CDF) files, including richer metadata compliant with the 
International Solar Terrestrial Program (ISTP) specifications;
\item Observation quicklooks in PDF format for both polarisation.
\end{itemize}

The NDA data is also available through a \emph{Das2} \citep{10.1002/essoar.10500359.1} interface, which 
is a data streaming interface for time dependent data. The Das2 server is providing server-side 
resampling as configured in the query of a Das2 client (e.g., Autoplot; \cite{10.1007/s12145-010-0049-0}). 
The Das2 endpoint for NDA data is accessible in the EPN-TAP service response, through the Datalink interface.
 
\subsubsection{NRH and ORFEES}
NRH and ORFEES are two instruments dedicated to the radio monitoring of the Sun, with applications to 
Space Weather. The NRH is a T-shaped interferometer composed of 47 antennas, and observing between 150 
MHz and 450 MHz, at a spatial resolution of about 2.5 arc-minutes. High temporal resolution data is 
available, as well as low resolution 2D images. The ORFEES instrument is a spectrograph coupled with a 
5m diameter antenna, observing the Sun from 130 MHz to 1 GHz, with a 100 ms temporal resolution. The 
data are accessible from the Radio Solar Data Base portal (\url{https://rsdb.obs-nancay.fr}) with an 
interactive interface. Two EPN-TAP services are developed (one for each instrument). The NRH and ORFEES data files available through EPN-TAP are:
\begin{itemize}
\item ORFEES dynamic spectra (FITS format)
\item NRH image data files (raw binary format)
\item NRH image data files (animated GIF files)
\end{itemize}


\subsubsection{NenuFAR}
NenuFAR (\url{https://nenufar.obs-nancay.fr/en/astronomer/}) is a new radioastronomy instrument, operating in the 10 to 85 MHz frequency band. Data can be acquired through four distinct observing modes, which rely on specific receivers and instrument configurations. An Obs-TAP service will be provided for the 'Standalone Beamformer' mode. In this operating configuration, NenuFAR acts as a low-frequency phased array instrument, beamforming data from 96 sub-arrays called 'Mini-Arrays'. The latter are consisting of 19 analog-phased crossed-dipoles gathered in hexagonal tiles, each of them rotated with respect to the others. Dynamic Spectrum data computation is achieved thanks to the UnDySPuTeD backend, with time and frequency resolutions ranging from 0.30 to 84.00 ms and from 0.10 to 12.20 kHz respectively. Reduced data (cleaned of radio-frequency interferences, corrected from instrument systematics, averaged in time/frequency and/or selected over given time periods or frequency bands) will be distributed, either in FITS or HDF5 format (with LOFAR compliant content). Quicklooks will also be delivered in PDF format.







%\begin{figure}[h!]
%\centering
%\includegraphics[scale=1.7]{universe}
%\caption{The Universe}
%\label{fig:universe}
%\end{figure}




\end{document}
